\documentclass{article}
% Change "article" to "report" to get rid of page number on title page
\usepackage{amsmath,amsfonts,amsthm,amssymb}
\usepackage{setspace}
\usepackage{Tabbing}
\usepackage{fancyhdr}
\usepackage{lastpage}
\usepackage{extramarks}
\usepackage{chngpage}
\usepackage{soul,color}
\usepackage{graphicx,float,wrapfig}
\usepackage{enumitem}

% In case you need to adjust margins:
\topmargin=-0.45in      %
\evensidemargin=0in     %
\oddsidemargin=0in      %
\textwidth=6.5in        %
\textheight=9.0in       %
\headsep=0.25in         %

% Homework Specific Information
\newcommand{\hmwkTitle}{Homework\ \#0}
\newcommand{\hmwkDueDate}{Friday,\ September\ 8,\ 2017}
\newcommand{\hmwkClass}{CS\ 585}
\newcommand{\hmwkClassTime}{}
\newcommand{\hmwkClassInstructor}{}
\newcommand{\hmwkAuthorName}{Richard\ Cui}

% Setup the header and footer
\pagestyle{fancy}                                                       %
\lhead{\hmwkAuthorName}                                                 %
\chead{\hmwkClass : \hmwkTitle}  %
\rhead{}                                                     %
\lfoot{\lastxmark}                                                      %
\cfoot{}                                                                %
\rfoot{Page\ \thepage\ of\ \pageref{LastPage}}                          %
\renewcommand\headrulewidth{0.4pt}                                      %
\renewcommand\footrulewidth{0.4pt}                                      %

% This is used to trace down (pin point) problems
% in latexing a document:
%\tracingall

%%%%%%%%%%%%%%%%%%%%%%%%%%%%%%%%%%%%%%%%%%%%%%%%%%%%%%%%%%%%%
% Some tools
\newcommand{\enterProblemHeader}[1]{\nobreak\extramarks{#1}{#1 continued on next page\ldots}\nobreak%
                                    \nobreak\extramarks{#1 (continued)}{#1 continued on next page\ldots}\nobreak}%
\newcommand{\exitProblemHeader}[1]{\nobreak\extramarks{#1 (continued)}{#1 continued on next page\ldots}\nobreak%
                                   \nobreak\extramarks{#1}{}\nobreak}%

\newlength{\labelLength}
\newcommand{\labelAnswer}[2]
  {\settowidth{\labelLength}{#1}%
   \addtolength{\labelLength}{0.25in}%
   \changetext{}{-\labelLength}{}{}{}%
   \noindent\fbox{\begin{minipage}[c]{\columnwidth}#2\end{minipage}}%
   \marginpar{\fbox{#1}}%

   % We put the blank space above in order to make sure this
   % \marginpar gets correctly placed.
   \changetext{}{+\labelLength}{}{}{}}%

\setcounter{secnumdepth}{0}
\newcommand{\homeworkProblemName}{}%
\newcounter{homeworkProblemCounter}%
\newenvironment{homeworkProblem}[1][Problem \arabic{homeworkProblemCounter}]%
  {\stepcounter{homeworkProblemCounter}%
   \renewcommand{\homeworkProblemName}{#1}%
   \section{\homeworkProblemName}%
   \enterProblemHeader{\homeworkProblemName}}%
  {\exitProblemHeader{\homeworkProblemName}}%

\newcommand{\problemAnswer}[1]
  {\noindent\fbox{\begin{minipage}[c]{\columnwidth}#1\end{minipage}}}%

\newcommand{\problemLAnswer}[1]
  {\labelAnswer{\homeworkProblemName}{#1}}

\newcommand{\homeworkSectionName}{}%
\newlength{\homeworkSectionLabelLength}{}%
\newenvironment{homeworkSection}[1]%
  {% We put this space here to make sure we're not connected to the above.
   % Otherwise the changetext can do funny things to the other margin

   \renewcommand{\homeworkSectionName}{#1}%
   \settowidth{\homeworkSectionLabelLength}{\homeworkSectionName}%
   \addtolength{\homeworkSectionLabelLength}{0.25in}%
   \changetext{}{-\homeworkSectionLabelLength}{}{}{}%
   \subsection{\homeworkSectionName}%
   \enterProblemHeader{\homeworkProblemName\ [\homeworkSectionName]}}%
  {\enterProblemHeader{\homeworkProblemName}%

   % We put the blank space above in order to make sure this margin
   % change doesn't happen too soon (otherwise \sectionAnswer's can
   % get ugly about their \marginpar placement.
   \changetext{}{+\homeworkSectionLabelLength}{}{}{}}%

\newcommand{\sectionAnswer}[1]
  {% We put this space here to make sure we're disconnected from the previous
   % passage

   \noindent\fbox{\begin{minipage}[c]{\columnwidth}#1\end{minipage}}%
   \enterProblemHeader{\homeworkProblemName}\exitProblemHeader{\homeworkProblemName}%
   \marginpar{\fbox{\homeworkSectionName}}%

   % We put the blank space above in order to make sure this
   % \marginpar gets correctly placed.
   }%

%%%%%%%%%%%%%%%%%%%%%%%%%%%%%%%%%%%%%%%%%%%%%%%%%%%%%%%%%%%%%


%%%%%%%%%%%%%%%%%%%%%%%%%%%%%%%%%%%%%%%%%%%%%%%%%%%%%%%%%%%%%
% Make title
\title{\vspace{2in}\textmd{\textbf{\hmwkClass:\ \hmwkTitle}}\\\normalsize\vspace{0.1in}\small{Due\ on\ \hmwkDueDate}\\\vspace{0.1in}\large{\textit{\hmwkClassInstructor\ \hmwkClassTime}}\vspace{3in}}
\date{}
\author{\textbf{\hmwkAuthorName}}
%%%%%%%%%%%%%%%%%%%%%%%%%%%%%%%%%%%%%%%%%%%%%%%%%%%%%%%%%%%%%

\begin{document}
\begin{spacing}{1.1}
\maketitle
\newpage
% Uncomment the \tableofcontents and \newpage lines to get a Contents page
% Uncomment the \setcounter line as well if you do NOT want subsections
%       listed in Contents
%\setcounter{tocdepth}{1}
%\tableofcontents
%\newpage

% When problems are long, it may be desirable to put a \newpage or a
% \clearpage before each homeworkProblem environment

\newpage
\begin{homeworkProblem} [\arabic{homeworkProblemCounter} Domain of a joint distribution]

\begin{homeworkSection}{1.1}
The joint distribution $P(A, B)$ defines probabilities for $4 \times 3 = 12$ possible outcomes.
\end{homeworkSection}

\begin{homeworkSection}{1.2}
The joint distribution $P(A_1, A_2, \ldots A_n)$ defines probabilities for $2^n$ possible outcomes.
\end{homeworkSection}

\end{homeworkProblem}

\clearpage
\begin{homeworkProblem} [\arabic{homeworkProblemCounter} Independence versus Basic Definitions]

\begin{homeworkSection}{2.1}
Which of the following statements is always true?

\begin{enumerate}
\item $P(A|B) = P(B|A)$ False
\item $P(A,B) = P(A|B)P(B)$ True
\item $P(A,B) = P(A)P(B)$ False
\item $P(A|B) = P(A)$ False
\item $P(A,B,C) = P(A)P(C)$ False
\item $P(A,B,C) = P(A)P(B)P(C)$ False
\item $P(A,B,C) = P(A)P(B|A)P(C|A,B)$ True
\item $P(A) = \sum_{b \in \text{domain}(B)} P(A, B=b)$ True
\item $P(A) = \sum_{b \in \text{domain}(B)} P(A | B=b) P(B=b)$ True
\end{enumerate}

\noindent Now assume that $A$, $B$, and $C$ are all independent of each other.
Which of these statements is true?

\begin{enumerate}
\item $P(A|B) = P(B|A)$ False
\item $P(A,B) = P(A|B)P(B)$ True
\item $P(A,B) = P(A)P(B)$ True
\item $P(A|B) = P(A)$ True
\item $P(A,B,C) = P(A)P(C)$ False
\item $P(A,B,C) = P(A)P(B)P(C)$ True
\item $P(A,B,C) = P(A)P(B|A)P(C|A,B)$ True
\item $P(A) = \sum_{b \in \text{domain}(B)} P(A, B=b)$ True
\item $P(A) = \sum_{b \in \text{domain}(B)} P(A | B=b) P(B=b)$ True
\end{enumerate}

\end{homeworkSection}

\end{homeworkProblem}

\clearpage
\begin{homeworkProblem} [\arabic{homeworkProblemCounter} Logarithms]

\begin{homeworkSection} {3.1 Log-probs}
The range of possible values for $\log(p)$ is $(-\infty, 0]$.
\end{homeworkSection}

\begin{homeworkSection} {3.2 Prob ratios}
The range of possible values for $\frac{p}{q}$ is $[0, \infty)$.
\end{homeworkSection}

\begin{homeworkSection} {3.3 Log prob ratio}
The range of possible values for $\log(\frac{p}{q})$ is $(-\infty, \infty)$.
\end{homeworkSection}


\end{homeworkProblem}

\clearpage
\begin{homeworkProblem} [\arabic{homeworkProblemCounter} Deriving Bayes Rule]
\begin{align*}
	P(A,B) = P(A)P(A|B) && \text{Definition of conditional probablity}\\
	P(B,A) = P(B)P(B|A) && \text{Definition of conditional probablity}\\
	P(A,B) = P(B,A) && \text{Commutative property}\\
	P(A)P(A|B) = P(B)P(B|A) && \text{Substitution}\\
	P(B|A) = \frac{P(A)P(A|B)}{P(B)} && \text{Divide by } P(B)
\end{align*}

\end{homeworkProblem}

\end{spacing}
\end{document}

%%%%%%%%%%%%%%%%%%%%%%%%%%%%%%%%%%%%%%%%%%%%%%%%%%%%%%%%%%%%%

%----------------------------------------------------------------------%
% The following is copyright and licensing information for
% redistribution of this LaTeX source code; it also includes a liability
% statement. If this source code is not being redistributed to others,
% it may be omitted. It has no effect on the function of the above code.
%----------------------------------------------------------------------%
% Copyright (c) 2007, 2008, 2009, 2010, 2011 by Theodore P. Pavlic
%
% Unless otherwise expressly stated, this work is licensed under the
% Creative Commons Attribution-Noncommercial 3.0 United States License. To
% view a copy of this license, visit
% http://creativecommons.org/licenses/by-nc/3.0/us/ or send a letter to
% Creative Commons, 171 Second Street, Suite 300, San Francisco,
% California, 94105, USA.
%
% THE SOFTWARE IS PROVIDED "AS IS", WITHOUT WARRANTY OF ANY KIND, EXPRESS
% OR IMPLIED, INCLUDING BUT NOT LIMITED TO THE WARRANTIES OF
% MERCHANTABILITY, FITNESS FOR A PARTICULAR PURPOSE AND NONINFRINGEMENT.
% IN NO EVENT SHALL THE AUTHORS OR COPYRIGHT HOLDERS BE LIABLE FOR ANY
% CLAIM, DAMAGES OR OTHER LIABILITY, WHETHER IN AN ACTION OF CONTRACT,
% TORT OR OTHERWISE, ARISING FROM, OUT OF OR IN CONNECTION WITH THE
% SOFTWARE OR THE USE OR OTHER DEALINGS IN THE SOFTWARE.
%----------------------------------------------------------------------%
